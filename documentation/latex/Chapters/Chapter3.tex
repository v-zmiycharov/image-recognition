\chapter{Използвани технологии и платформи}

\label{Chapter3}

%----------------------------------------------------------------------------------------

\section{Обучителни данни}

ImageNet е огромна база от данни с изображения, предназначена за използване от разработчици на софтуери за разпознаване на изображения. От 2016 година, повече от 10 милиона изображения са анотирани ръчно от ImageNet, за да покажат какви обекти са показани. Очертаващи линии на обектите в изображенията са дадени за над 1 милион картинки \cite{ImageNet}. Базата с изображения, както и техните анотации е свободно достъпна в ImageNet, но те не са тяхно притежание.

Процесът на анотация се извършва чрез външни хора. Анотациите на ниво изображение индикират наличието или липсата на даден клас в картинката. Анотациите, които са за конкретен обект от картинката предоставят и ограждащи граници на обекта (видимата му част върху изображението). ImageNet използва йерархична словесна схема за организация като поддържа клас куче и над 120 под-класа, които са различни породи кучета. Това позволява избор на разработчиците колко прецизно и на какво ниво на абстракция искат да е класифицирането.

С цел обучение бяха изтеглени 57 класа животни, съдържащи между 1500 и 2000 изображения за клас. Те са с различни размери и обхващат разнообразни фонове и среди. Изтеглените класове са показани на Фигура \ref{tab:tableclasses}.

\begin{longtable}{ | c | c | c | }
\hline
\textbf{Номер} & \textbf{Име} & \textbf{Идентификатор} \\ \hline \hline
0 & alligator & n01698434 \\ \hline
1 & antelope & n02419796 \\ \hline
2 & bat & n02139199 \\ \hline
3 & bear & n02131653 \\ \hline
4 & bison & n02410509 \\ \hline
5 & butterfly & n02274259 \\ \hline
6 & camel & n02437136 \\ \hline
7 & canary & n01533339 \\ \hline
8 & cat & n02121808 \\ \hline
9 & cattle & n04052442 \\ \hline
10 & chicken & n01791625 \\ \hline
11 & cockroach & n02233338 \\ \hline
12 & deer & n02430045 \\ \hline
13 & dog & n02084071 \\ \hline
14 & donkey & n02389559 \\ \hline
15 & duck & n01846331 \\ \hline
16 & eagle & n01613294 \\ \hline
17 & elephant & n02503517 \\ \hline
18 & fly & n02190166 \\ \hline
19 & fox & n02118333 \\ \hline
20 & frog & n01641739 \\ \hline
21 & giraffe & n02439033 \\ \hline
22 & goat & n02416880 \\ \hline
23 & goldfish & n01443537 \\ \hline
24 & horse & n02374451 \\ \hline
25 & hyena & n02116738 \\ \hline
26 & iguana & n01677366 \\ \hline
27 & jaguar & n02128925 \\ \hline
28 & kangaroo & n01877134 \\ \hline
29 & lion & n02129165 \\ \hline
30 & llama & n02437616 \\ \hline
31 & monkey & n02484322 \\ \hline
32 & mosquito & n02200198 \\ \hline
33 & mouse & n02330245 \\ \hline
34 & octopus & n01970164 \\ \hline
35 & ostrich & n01518878 \\ \hline
36 & owl & n01621127 \\ \hline
37 & panda & n02510455 \\ \hline
38 & parrot & n01817346 \\ \hline
39 & penguin & n02055803 \\ \hline
40 & pig & n02395406 \\ \hline
41 & pigeon & n01814921 \\ \hline
42 & piranha & n02584449 \\ \hline
43 & rabbit & n02324587 \\ \hline
44 & raven & n01579260 \\ \hline
45 & scorpion & n01770393 \\ \hline
46 & shark & n01482330 \\ \hline
47 & sheep & n10588074 \\ \hline
48 & snail & n01944390 \\ \hline
49 & snake & n01726692 \\ \hline
50 & spider & n01772222 \\ \hline
51 & swan & n01858441 \\ \hline
52 & tiger & n02129604 \\ \hline
53 & turkey & n01794344 \\ \hline
54 & whale & n02062744 \\ \hline
55 & worm & n01922303 \\ \hline
56 & zebra & n02391049 \\ \hline
\caption{Обучително множество от класове}
\label{tab:tableclasses}
\end{longtable}

\section{Технологии, платформи и методологии}
Една от основните причини разпознаването на изображение да бъде възможно е напредъкът на хардуера и възможността да използваме мощността и бързия достъп до паметта на видео картите (GPU). Един от най-удобните начини това да се случи се предоставя от CUDA \cite{CUDA}. Като програмен език е избран Python \cite{Python}, който е най-популярното решение в тази сфера, защото предоставя лесна интеграция с интерфейси за достъп до видео картата и има много популярни и добри библиотеки за изкуствен интелект. За изграждане на конволюционните невронни мрежи се използва TensorFlow \cite{Tensorflow}. За управление на версиите се разчита на скрито Git хранилище в Bitbucket \cite{Bitbucket}.

\subsection{CUDA}
CUDA е платформа за паралелна обработка и програмен интерфейс (API), създаден от Nvidia. Той позволява на софтуерните разработчици да използват видео карти, които позволяват CUDA за обработка с каквато и да е цел - този подход се нарича GPGPU (General-Purpose computing on Graphics Processing Units). CUDA платформата е софтуерен слой, който позволява директен достъп до видео картата и елементите за паралелна обработка за всякакви изчислителни цели. \cite{EndOfCPU}

CUDA е предназначен за работа с програмни езици като C,C++ и Фортран. Тази достъпност прави по-лесен за специалистите по паралелна обработка да използват ресурсите на видео картата за разлика от други услуги като Direct3D и OpenGL, които изискват добри умения в графичното програмиране. Също така CUDA поддържа програмни среди като OpenACC и OpenCL. Когато за първи път е представен от Nvidia, CUDA е бил абревиатура на Compute Unified Device Architecture, но впоследствие Nvidia се отказва от съкращението.

CUDA предоставя услуги както на ниско, така и на високо ниво на абстракция. Първата услуга CUDA пуска в публичното пространство на 15.02.2007г. с поддръжка на Windows и Linux. Mac OS X поддръжка се добавя във версия 2.0. CUDA работи с всички видео карти на Nvidia и повечето стандартни операционни системи.

\subsection{Python}

Python е глобално използван динамичен език за програмиране от високо ниво \cite{Python_1}. Философията на неговия дизайн набляга на четимост на кода и неговият синтаксис позволява на програмистите да имплементират концепции с по-малко линии код в сравнение с други езици като C++ и Java. Езикът предоставя конструкции, които са предназначени за писане на чист код както за малки, така и за големи програми. \cite{Python_2}

Python поддържа множество програмни парадигми като обектно-ориентирано, императивно, функционално програмиране и процедурни стилове. При него има динамично типизиране и автоматично управление на паметта. Има много и качествени налични библиотеки.

Python интерпретатори са налични за много операционни системи, което позволява да се изпълнява Python код при почти всякакви системи. CPython, както и повечето различни имплементации на Python са с отворен код.

\subsubsection{Структура и функционалност}
Python предлага добра структура и поддръжка за разработка на големи приложения. Той притежава вградени сложни типове данни като гъвкави масиви и речници, за които биха били необходими дни, за да се напишат ефикасно на C.

Python позволява разделянето на една програма на модули, които могат да се използват отново в други програми. Също така притежава голям набор от стандартни модули, които да се използват като основа на програмите. Съществуват и вградени модули, които обезпечават такива неща като файлов вход/изход (I/O), различни системни функции, сокети (sockets), програмни интерфейси към GUI-библиотеки като Тк, както и много други.

Тъй като Python е език, който се интерпретира, се спестява значително време за разработка, тъй като не са необходими компилиране и свързване (linking) за тестването на дадено приложение. Освен това, бидейки интерпретируем език с идеология сходна с тази на Java, приложение, написано на него, е сравнително лесно преносимо на множеството от останали платформи (или операционни системи).

Програмите, написани на Python, са доста компактни и четими, като често те са и по-кратки от еквивалентните им, написани на C/C++. Това е така, тъй като:

\begin{itemize}
\item наличните сложни типове данни позволяват изразяването на сложни действия с един-единствен оператор;
\item групирането на изразите се извършва чрез отстъп, вместо чрез начални и крайни скоби или някакви други ключови думи (друг език, използващ такъв начин на подредба, е Haskell);
\item не са необходими декларации на променливи или аргументи.
\item Python съдържа прости конструкции, характерни за функционалния стил на програмиране, които му придават допълнителна гъвкавост
\end{itemize}

Всеки модул на Python се компилира преди изпълнение до код за съответната виртуална машина. Този код се записва за повторна употреба като .pyc файл.

Програмите написани на Python представляват съвкупност от файлове с изходен код. При първото си изпълнение този код се компилира до байткод, а при всяко следващо се използва кеширана версия. Байткодът се изпълнява от интерпретатор на Python.

\begin{itemize}
\item Строго типизиран (strong typing) – При несъответствие между типовете е необходимо изрично конвертиране.
\item Динамично типизиран (dynamic typing) – Типовете на данните се определят по време на изпълнението. Работи на принципа duck typing – Оценява типа на обектите според техните свойства.
\item Използва garbage collector – вътрешната реализация на езика се грижи за управлението на паметта.
\item Блоковете се формират посредством отстъп. Като разграничител между програмните фрагменти използва нов ред.
\end{itemize}

\subsection{Tensorflow}

Tensorflow e софтуерна библиотека с отворен код за числови изчисления с помощта на графи за предаване на данните (data flow graphs). Всеки връх в графа представлява математически операции, докато пътищата са многомерни масиви (тенсори - tensors), които служат за комуникация между тях. Архитектурата позволява да се извършват изчисления на повече от един процесор или видео карта както на десктоп, така и на сървър или мобилно устройство използвайки един единствен програмен интерфейс (API). Tensorflow първоначално е разработен от изследователи и инженери, работещи в Google по проекта Google Brain \cite{GoogleBrain} по разработка за изследователската организация Google's Machine Intelligence. Първоначалната им цел е била изследване в сферите на машинното самообучение и невронните мрежи, но системата е достатъчно генерална, за да се прилага свободно и в други области. \cite{Tensorflow}

\subsubsection{Въведение}

Проектът Google Brain стартира през 2011, за да изследва използването на дълбоки невронни мрежи с много големи мащаби както с изследователска цел така и за да се използва от Google. Като част от началната работа по този проект първо е построен DistBelief, система от първа генерация за скалируемо трениране. Благодарение на DistBelief са извършени множество изследователски дейности като обучение без надзор, репрецентация на езици, модели за класифициране на изображения, разпознаване на обекти, класификация на видеа, разпознаване на реч, игра на Go, разпознаване на граждани и множество други сфери \cite{TensorflowDetails}. Освен това често след взаимодействие с екипът Google Brain, повече от 50 други екипи от Google и други комнпании са използвали дълбоки невронни мрежи с помощта на DistBelief в много продукти, сред които Google Search, рекламни продукти, системи за разпознаване на реч, Google Photos, Google Maps и StreetView, Google Translate, YouTube и много други. 

На база на опита, придобит от DistBelief и след по-добро разбиране на това какво е нужно да има такава система е създаден Tensorflow - система от втора генерация за имплементиране и използване на модели за машинно самообучение от големи мащаби. Tensorflow чете изчисленията с модел, който използва поток от данни и ги нагажда спрямо голямо разнообразие от хардуерни платформи. Той може да се използва както от мобилни устройства с Android или iOS, така и от отделни машини с една или повече видео карти. Налична е поддръжката и на паралелна работа върху много машини. Разполагайки с една система, която може да се изпълнява върху голям набор от хардуерни платформи значително опростява използването на системи за машинно самообучение. Изчисленията в Tensorflow са представени като графи на данните със състояния. Той позволява бързи експерименти с нови модели за изследователски цели и предоставя много висока производителност. С цел да се скалира тренирането на невронни мрежи за по-големи системи, Tensorflow позволява лесно да се изградят паралелни операции на различни машини, които обновяват множество споделени параметри и състояния. Лесният за употреба интерфейс позволява да се постигат различни техники за паралелна обработка с много малко усилия. Някои употреби на Tensorflow позволяват гъвкавост по отношение на цялостност при обновяването на параметрите и това много лесно може да се използва при големи системи. В сравнение с DistBelief, програмният модел на Tensorflow е по-гъвкав, неговата производителност е значително по-добра и поддържа използване на по-голямо множество от модели върху повече хардуерни платформи.

\subsubsection{Програмен модел и основни концепции}

Изчисленията в Tensorflow представляват насочен граф, който е изграден от множество върхове. Графът представлява изчисление на поток от данни с разширения, които позволяват някои пътища да не се променят и дават контрол вътху разклоненията и циклите на пътищата в графа. Най-честите употреби на Tensorflow са чрез езиците за програмиране C++ и Python.

В графа на TensorFlow всеки връх има 0 или повече входа и 0 или повече изхода и представлява инстанция на една операция. Стойности, които се пренасят по стандартните пътища в графа се наричат тензори. Те са масиви с променливи измерения, при които типът на данните се определя при създаване на графа. Има и специални пътища, които контролират зависимостите. През тях не се предават данни, но те синхронизират работата на останалите пътища и определят кога дадена операция да започне да се изпълнява. Tensorflow понякога автоматично добавя такива специални пътища, за да въведе ред над независими до преди този момент операции. Това се прави с различни цели като например оптимизирано използване на паметта.\\

\textbf{Операции и ядра}

Всяка операция има име и представлява абстракция на изчисление (например умножение на матрици или събиране). Операциите могат да имат атрибути и всички атрибути трябва да бъдат дефинирани по време на изграждане на графа, за да се създаде връх в него, който да отговаря за операцията. Една от честите употреби на атрибутите е една операция да се направи полиморфна върху различни типове данни.

Под ядро се разбира конкретна имплементация на операция, която може да се изпълни върху конкретен вид устройство (например процесор или видео карта). Tensorflow дефинира видовете операции и ядра посредством механизъм за регистрация. Това множество може да се разширява чрез свързване на допълнителни дефиниции на операции и/или ядра.\\

\textbf{Сесии}

Всяка клиентска програма взаимодейства със системата на Tensorflow като създава нова сесия. С цел да създаде граф на изчисленията, интерфейсът на сесията поддържа метод за разширение, за да разшири графа с допълнителни върхове и пътища (при създаване на сесията графът е празен). Другата основна операция, която се поддържа от интерфейса на сесията е "Стартирай" (Run), която приема множество изходи, които трябва да се изчислят, както и множество тензори, които могат да се добавят към графа. В повечето случаи на употреба на Tensorflow след като веднъж се създаде сесия с граф, върху нея се извършват хиляди или милиони стартирания.\\

\textbf{Операция Variable}

При повечето изчисления един граф се използва много пъти. Повечето тензори не оцеляват по време на едно изпълнение на графа. Variable обаче е специален вид операция, която генерира връзка към постоянен непроменящ се тензор, който се запазва след няколко изпълнения върху графа. Тези връзки мгоат да бъдат предадени към специални операции като Assign или AssignAdd, които изменят съответния тензор. За приложенията на Tensorflow, свързани с машинно самообучение, параметрите на модела обикновено се пазят в тензори, които се пазят в променливи (Variable) и се обновяват, когато се стартира тренировъчния граф за модела.\\

\textbf{Алтернативи}

Theano също е библиотека в Python за цифрови изчисления. При нея те са представени с помощта на Numpy синтаксис \cite{Theano} и могат да се използват както върху процесор, така и върху видео карта. Theano е с отворен код и е основно разработен от група за машинно самообучение от Университета на Монреал.

Caffe е библиотека, написана на C++. Първо е започната като докторантски проект в UC Berkeley, след което става с отворен код и за развитието ѝ допринасят много програмисти.

Има и други алтернативи на Tensorflow като Torch, CNTK (Computational Network Toolkit) на Microsoft, DSSTNE (Deep Scalable Sparse Tensor Network Engine) на Amazon.

\subsection{Keras}

Keras е библиотека за невронни мрежи от високо ниво. Написана е на Python и може да използва Tensorflow или Theano. Първоначалната ѝ цел е била да позволи бързи и лесни експерименти. Философията на създателите е, че възможността да стигнеш от идея до резултат за възможно най-кратко време е ключът към успешните изследователски дейности.

Сред основните предимства на библиотеката са:
\begin{itemize}
\item Модулярност: Слоевете на невронната мрежа, оптимизаторите, активационните функции, схемите за регуларизация са независими модули, които могат да се комбинират, за да създадат нов модул.
\item Опростена имплементация: Всеки модул е кратък и лесен за разбиране. Всеки ред код е интуитивен за четене още при първия преглед на имплементацията.
\item Възможност за разширяване: Много е лесно да се добавят нови модули, а за вече съществуващите са предоставени множество примери. Това прави Keras подходящ и за по-сложни изследователски цели.
\item Директна работа с Python. Няма нужда да се добавят допълнителни модули или конфигурационни файлове.
\end{itemize}

\subsection{Git}

Git е система за управление на версиите, следене на промените на компютърни файлове и координиране на работата по тези файлове с други хора. Той основно се използва за софтуерна разработка, но може да се използва и за други цели. Основните му приоритети са скорост, интегритет на данните и поддръжка на дистрибутирани, нелинейни процеси.

Git е създаден от Линус Торвалдс през 2005 година, за да се разработи ядрото на Линукс заедно с други разработчици, които допринасят за първоначалния му напредък.

Както и повечето други системи за управление на версиите (и за разлика от повечето системи с начин на комуникация клиент-сървър), всяка Git директория на всеки компютър съдържа в себе си цялата информация и история за хранилището. Това позволява пълно следене на версиите, независимо от наличието на мрежа или централен сървър.

