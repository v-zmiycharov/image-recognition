\chapter{Приложения}

\label{Applications}

%----------------------------------------------------------------------------------------

\section{Приложение 1: Указание за инсталиране}

Предварителни изисквания:

\begin{itemize}
\item Linux базирана операционна система (софтуерът е тестван под Ubuntu 14.04 и Ubuntu 16.04).
\item Python версия 3 или по-нова.
\item Видео карта NVIDIA - интерфейсът CUDA разпознава само видео карти на NVIDIA. Ако видео картата не е NVIDIA софтуерът ще използва мощността на процесора и системата ще се забави значително.
\end{itemize}

Софтуерът използва Tensorflow, който от своя страна достъпва видео картата чрез CUDA. За целта трябва да се инсталират (Подробно описание: https://www.tensorflow.org/versions/r0.11/get\_started/os\_setup):

\begin{itemize}
\item CUDA toolkit 8.0
\item CuDNN v5
\item Tensorflow
\end{itemize}

Повечето модули, от които зависят програмите могат да се инсталират чрез pip. В ubuntu по подразбиране са инсталирани 2 версии на python: 2 и 3. За да се инсталира пакет с помощта на pip за python3 се използва pip3. Някои модули изискват повече права от други - трябва да се добави sudo пред pip3 за работа като администратор (su=super user). Python имплементацията зависи от следните модули:

\begin{itemize}
\item \textbf{os}: Използва се за записване и четене на файлове. Наличен е по подразбиране в python.
\item \textbf{random}: Използва се за генериране на произволни числа. Наличен е по подразбиране в python.
\item \textbf{pickle}: Използва се за сериализация (запазване на текущото състояние на обекти във файл). Наличен е по подразбиране в python. 
\item \textbf{time}: Използва се за работа с дати и часове. Наличен е по подразбиране в python.
\item \textbf{tarfile}: Използва се за работа с tar файлове. В такъв формат се свалят първоначално снимките от ImageNet. Наличен е по подразбиране в python.
\item \textbf{multiprocessing}: Използва се паралелна работа в повече от една нишка. Ускорява изтеглянето на изображенията и други процеси. Наличен е по подразбиране в python.
\item \textbf{pprint}: Използва се за принтиране на сложни структури от данни като речници и списъци. Наличен е по подразбиране в python.
\item \textbf{numpy}: Този модул е в основата на всички научни разработки, написани на python. Поддържа оптимизирани многомерни масиви, обработката им чрез функции от високо ниво и полезни практики от линейната алгебра. Ако не е наличен се инсталира с "pip install numpy".
\item \textbf{PIL}: Абревиатура на Python Imaging Library. Използва се за обработка на изображения. В контекста на програмата помага за преоразмеряването на всяко изображение в нужния формат. Инсталира се с "pip install pillow".

\item \textbf{h5py}: Използва се за запазване на вече обучен модел на невронна мрежа. Инсталира се последователно със следните 2 команди:

sudo apt-get install libhdf5\\
sudo pip3 install h5py
\item \textbf{keras}: Абстракция на работата с невронни мрежи. Използва Tensorflow. Инсталира се с "pip install keras". За повече информация: https://keras.io/\#installation.
\end{itemize}

\section{Приложение 2: Наръчник на потребителя}


Файловата организация е както следва:

\begin{itemize}
\item src
	\begin{itemize}
      \item data
      	\begin{itemize}
          \item bin\_generator.py
          \item image\_crawler.py
          \item imagenet\_metadata.py
        \end{itemize}
      \item main
      	\begin{itemize}
          \item common.py
          \item eval.py
          \item models.py
          \item retrain.py
          \item train.py
        \end{itemize}
      \item pretrained
      	\begin{itemize}
          \item audio\_conv\_utils.py
          \item imagenet\_utils.py
          \item inception\_v3.py
          \item resnet50.py
          \item test\_imagenet.py
          \item vgg16.py
          \item vgg19.py
          \item xception.py
        \end{itemize}
      \item config.py
    \end{itemize}
\item definitions.py
\item main.py
\end{itemize}


Единствената изходна точка на програмата е файлът \textbf{main.py} в директорията /image-recognition. След успешно стартиране на програмата, тя предоставя 4 опции на потребителя в следния диалог:\\
1. Image crawler\\
2. Bin Generator\\
3. Train\\
4. Eval\\
Choose what to do:

При избор на първата опция (\textbf{1. Image crawler}), програмата изтегля изображенията за 57-те класа в папка \_DATA в основната директория. Процесът може да отнеме време в зависимост от връзката към интернет и ядрата на компютъра. След изтегляне на всички изображения папката заема около 8.5 GB дисково пространство.

При избор на втората опция (\textbf{2. Bin Generator}), програмата създава папка в директорията \_MODELS с име текущата дата и час. Там добавя файлове с разширение .bin. Всеки един от тях съдържа информация за до 1000 изображения в удобен за четене формат. Броят на изображенията, за които е генерирана информация зависи от конфигурационния файл image-recognition/src/config.py.

При избор на третата опция (\textbf{3. Train}), програмата предоставя опция за избор на директория с генерирани .bin файлове. След избор, програмата тренира конволюционна невронна мрежа и запазва модела в същата папка в два файла: \_model.h5 и \_model.json. Освен това програмата принтира подробна информация за архитектурата на мрежата и точността при всяка итерация.

При избор на четвъртата опция (\textbf{4. Eval}), програмата отново предоставя опция за избор на директория с генериран модел. След избор, програмата оценява обучената невронна мрежа върху всеки един от класовете, за които е тренирана и изкарва статистика за точността.






