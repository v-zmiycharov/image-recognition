\chapter{Заключение}

\label{Chapter5}

%----------------------------------------------------------------------------------------

\section{Обобщение на изпълнението на началните цели}

Следва списък на началните цели и подробен анализ за постигнатите резултати относно всяка една от тях:
\begin{enumerate}
\item \textbf{Обзор на проблемната област}.\\
Областта е анализирана в детайли. Разгледани са различни приложения на класифицирането на изображения и бъдещето на тази сфера. В детайли са споменати някои от компаниите, които се занимават с това, както и по какво тяхното решение се отличава от останалите.

\item \textbf{Какви са условията (финансови, хардуерни и др.), за да построим класификатор с добра точност? Какво е добра точност и каква точност постигат готовите решения?}\\
Поради големият обем изображения, нужен за да се тренира добра конволюционна невронна мрежа хардуерът е от основно значение. Дори без да разполагаме с такъв можем да наемем cloud инстанция. С цел анализ, ускоряване на работата и постигане на по-добри резултати беше наета инстанция от Амазон. Тя успя да ускори работата 27-28 пъти в сравнение с локалната машина. Цената на услугата е в зависимост от ползването и варира: 25-40 цента на час за единично стартиране на инстанция и 90 цента на час за постоянна инстанция с гарантирана наличност. Амазон и други доставчици предлагат и по-мощни машини, за да удовлетворят всякакви цели.

Какво е добра точност е много относителен въпрос и зависи от начина, по който е зададен проблемът. Най-доброто известно решение до момента постига малко над 80\% точност за първият върнат клас (top 1). Ако проблемът се сведе до връщане на редица тагове за дадено изображение, то бихме могли да представим други мерки за точност, като например колко от върнатите тагове са адекватни за изображението и дали най-важните тагове са върнати от системата.

\item \textbf{Подбор на данни за обучение.}\\
Благодарение на ImageNet подборът на обучителни данни за целите на дипломната работа беше много бърз и лесен. Не така стоят нещата обаче, когато решението трябва да е глобално и да е обучено за всякакви теми и обекти. Някои компании изискват от клиентите си да предоставят тренировъчни изображения ако проблемът, който искат да се разреши е много специфичен.

\item \textbf{Реализация на прототип за класификация на изображения.}\\
Успешно беше реализиран прототип за класификация на изображения. За целите на дипломната работа са избрани класове, съставени от различни видове животни. Бяха изпробвани различни архитектури на конволюционни невронни мрежи и накрая трениран класификатор за разпонаване на 57 класа животни с 62920 тренировъчни изображения и 15731 тестови.

\item \textbf{Експерименти за оценка на точността на класификацията и възможностите за работа с големи данни.}\\
Направени са множество експерименти с 2 различни архитектури и техни разновидности (общо 5). Точността е измервана при всеки опит като 20\% от изображенията се заделят като тестови. С увеличение на броя на изображенията и с увеличение на техния размер времето за трениране и оценка нараства пропорционално. На наетата от Амазон инстанция при трениране на над 62000 изображения една итерация отнема 4 минути. На локалната машина тя би отнела над 1 час.

\item \textbf{Анализ на получените резултати.}\\
Резултатите са подробно анализирани. При 5 класа се постига точност над 63\% за тестовите и над 96\% за тренировъчните данни. При трениране върху 57 класа се постига точност над 34\% за тестовите и над 66\% за тренировъчните. При повечето опити е наличен проблемът за прекомерното нагаждане, който може да се обясни както с архитектурите и естеството на конволюционните невронни мрежи, така и с много разнообразните изображения от ImageNet.
\end{enumerate}

\section{Насоки за бъдещо развитие и усъвършенстване}
Изработената система може да бъде развивана и усъвършенствана в много насоки:

\begin{enumerate}
\item С цел постигане на по-добри резултати биха могли да се направят още множество експерименти, включително изпробване на други архитектури на невронната мрежа, промяна на хиперпараметрите, избор на нови тренировъчни изображения.
\item За по-бърза работа би могло да се наеме още по-добър хардуер и да се оптимизира процесът да използва повече от една видео карта. Това се позволява от интерфейса на библиотеката Tensorflow.
\item Системата би могла да се специализира в конкретна насока като например разпознаване на лица, идентифициране на туристически обекти или други.
\end{enumerate}
